%----------------------------------------------------------------------------
\chapter{\bevezetes}
%----------------------------------------------------------------------------

% A mai modern világban egyre elterjedtebbek a járműgyártók körében a különböző elektronikai megoldások alkalmazása járművek működtetésében. Nem kell feltétlenül az elektromos hajtású járművekre gondolni, amelyek nem mellesleg évről évre nagyobb darabszámban kerülnek gyártásra és eladásra, hanem sokkal inkább már az elmúlt 10-15 évben gyártott autókra is jellemző volt egyrészről a digitális műszerfal, kényelmi funkciók, ablaknyitó motor vagy az ülés állítására szolgáló elektronika.
% Ahogy haladunk előre az időben a járművek történetében úgy 

%%%

% Járműelektronika a kezdetekben: műszerfal, navigáció, elektromos ablakmotor, légkondicionáló

A járműelektronika, mint olyan elektronikus rendszer amely járművek belsejének valamelyik részén kerülnek integrálásra egy adott feladat ellátására, már a járművek történelmének korai korszakában is fellelhetőek. Ami a kezdetekben csak egy fedélzeti rádió volt az 1930-as években, az később bővült a gyújtási rendszer elektronikai alapokra helyezésével az 1960-as évektől, később pedig a technológia és a félvezetők fejlődésével egyre több és több feladatot láttak el. 

Ha csak belegondolunk, hogy milyen elektronikai eszközök lehetnek egy modern gépjárműben akkor hamar észrevesszük, hogy az ablaktörlő, az oldalsó ablakok és ülések mozgatása, környezetvédelmi szempontokból a motort szabályozó különböző érzékelők, kamerarendszerek vagy a napjaink prémium járműveiben már érintőképernyős fedélzeti számítógépek és akár a szervókormányok elektomos rásegítése mind ilyen elektronikai rendszerek használatával történnek.

Ezeket a beágyazott rendszereket általánosan elektronikai vezérlőegységeknek (ECU, electronic control unit) nevezzük.\\

% Technológiai forradalmak
%   Elektromos járművek: először hibrid majd teljes elektromos hajtáslánc, környezetvédelmi szempontok, áthelyezik a mechanikus rendszereket elektronikusra
%Az autóipar egy forradalmi megújuláson megy keresztül az elmúlt 10-20 évben. A legismertebb jelei ennek egyrészről az először hibrid majd ma már teljesen elektromos hajtásláncú járművek elterjedése, amelyekből bizonyos adatok szerint évente akár 50\%-kal több is kerülhet forgalomba. Ez a technológia egyrészről egy környezetvédelmi és széndioxid kibocsátás csökkentésére érkező megoldás, másrészről viszont ez tekinthető az elektronikai megoldások leglátványosabb felhasználásának a járművekben, amikor a hajtáslánc mint rendszer a mechanikus megoldások helyett egy teljesen villamossági alapra helyeződik át.

Az elmúlt 10-20 évben több technológiai forradalom is elindult az autóiparban, ebből az egyik az először hibrid, majd teljesen elektromos hajtásláncú járműveknek a drasztikus terjedése, bizonyos adatok szerint évente akár 50\%-kal több is kerülhet forgalomba. Ezek az elsősorban környezetvédelmi szempontok miatt, kőolaj és egyéb nemfenntartható energiaforrástól való elszakadást szolgálják. Ennek a forradalomnak egyik nagy eredménye ami az én témámhoz is kapcsolódik az a tendencia, miszerint már a legkomplexebb mechanikai folyamatokat amelyek egy járművet működtettek is villamossági alapokra terelték át, ezzel technikailag a járművekre mint komplex beágyazott rendszerekre is lehet tekinteni.

%   Mesterséges intelligencia: vezetéstámogató rendszerek, kényelmi funkciók és üzembiztonság (vonalkövetés, parkoló szenzorok), fék, elektromos meghajtású kormányrendszer, valamint előbb említett elektromos hajtáslánc, lassítás, sebességtartás stb.

% Másrészről aminek szintén nagy hatása van a modern járművek felépítésére és az azokban lévő technológiai megoldásokra az a mesterséges intelligencia fejlődése amely először a fejlett vezetéstámogató rendszerekben (advanced driving assistance system, röviden ADAS) volt fellelhető, de ma már szintén léteznek akár teljes önvezetésre is képes járművek. Ezeknek rendszereknek a feladata egyrészről kényelmi (vonalkövetés, tempo)

A másik nagy forradalom ami az én témámat érinti az a kezdetekben fejlett vezetéstámogató rendszerek (ADAS, advanced driver assistance systems), ma viszont már akár teljes önvezetésre képes járműveknek a megjelenése. Ez bizonyos szempontból összefügg az előzővel, hiszen a komplex jármű folyamatok átterelését villamossági alapokra, részben az önvezetésre való törekvésnek is köszönhető. Azonban érdemes kiemelni, hogy már az önvezetés nélkül is, a jármű évjáratától és felszereltségétől függően, találhatunk rengeteg vezetéstámogató rendszert. Ezek a rendszerek már olyan komplexitással rendelkeznek amiben a jármű különböző komponensei, amelyek akár külön beszállítóktól is érkezhetnek, koordinált feladatvégrehajtást is képesek ellátni. Ezeknek a rendszereknek a feladata lehet egyrészről kényelmi (pl. tempomat, parkolás segítő, stb.) vagy biztonsági (pl. vonalkövetés, holttérfigyelő, fáradtságérzékelő, stb.).

% INFO

Habár ezeknek a rendszereknek az ismertetése egy külön fejezetet is megérne, az amivel én foglalkozok, hogy ezeknek a rendszereknek a jelenléte és a elektronikai megoldásoknak terjedése egy olyan kockázatot von magával amelyre az autóipar és járműgyártás rugalmatlan struktúrái még viszonlyag limitáltan vannak felkészülve, ez pedig a kiberbiztonsági kockázatoknak a megjelenése. \\

% Modellalapú tervezése az E/E rendszereknek: Üzembiztonság, korai analízisek, terv alapú fejlesztés
% Modellalapú tervezése IT biztonsági rendszereknek: Fenyegetésmodellek, architektúrák, többrétegű védelem, kockázatalapú biztonság
% Modellalapú tervezése kiberbiztonsági rendszereknek az autóiparban: TARA, megvizsgált kutatások, safety és IT security keretrendszerek adaptálása

Habár a biztonsági kockázatok kezelése nem új dolog az autóiparban, hiszen az üzembiztonság már több mint egy évtizede szabványosítva (ISO 26262, 2011) működik, addig a kiberbiztonság még csak pár éve került szabványosításra (ISO 21434, 2021). Ez azt jelenti, hogy a kiberbiztonsági elemzése ezeknek a rendszereknek még sokkal kevesebb magas érettségű technikával rendelkezik, mint az üzembiztonsági elemzések.

Üzembiztonság esetén a kockázatok már korai fázisban felmérésre kerülnek és azokhoz a fejlesztési időt megelőzően, tervezési fázisban meghatározzák a mitigációkat. Ez annyit tesz, hogy már az egyes komponensek tervezésekor, lehetnek azok rendszer-, szoftver- vagy hardverszintű hibák, a kezdeti architektúra is úgy van meghatározva, hogy ezeknek a potenciális hibáknak az előfordulása minimális legyen.

Ezzel szemben a kiberbiztonság egy fiatal terület a kiber-fizikai rendszerek világában, viszont az is rendelkezik egy pár évtizedes múlttal az IT rendszereknél. Itt jellemzően már kész integrált rendszereknek történik az elemzése, felmérik a potenciális belépésipontjait egy támadónak, azoknak a lehetséges céljait, majd ezekre határoznak meg további szoftveres (pl. tűzfal) vagy hardveres (pl. DMZ) védelmeket, ezt a folyamatot nevezik általánosságban fenyegetésmodellezésnek (threat modelling). Még szintén fontos megemlíteni a monitorozás és utánkövetést, hiszen újabb és újabb sérülékenységek kerülhetnek elő a termék életciklusa során amelyeket utólag kell javítani ezeknél a rendszereknél.

Ezzel együtt is a korábban említett ISO 21434 szabvány tesz több ajánlást az IT rendszerek biztonsági elemzésére felhasznált módszerek adaptálására egy az üzembiztonsághoz hasonló kockázatalapú tervezési fázisú felmérésre, ezt nevezik úgy, hogy Threat Analysis and Risk Assessment (TARA).

Az én célom először egy olyan automatizmus készítése, amely az autóiparra már jellemző modellekből kiberbiztonsági elemzésre alkalmas modelleket készít, ez egyfajta származtatás lenne a rendszermodell és a fenyegetésmodell között. Majd ennek a származtatott modellnek az elemzésére fejlesztek egy eszközt, ami egyrészről követi az ISO 21434-ben definiált TARA követelményeit és ajánlásait, másrészről pedig felgyorsítja a kiberbiztonsági mérnökök munkáját támadási fák valamint dokumentumok automatikus generálásával.\\

A \textit{Háttérismeretek} fejezetben bemutatom az autóipari kiberbiztonsg szabályozási területét, bevezetem a szükséges fenyegetésmodellezési fogalmakat valamint bemutatom a megvalósításhoz használt eszközöket.

A \textit{Kapcsolódó tanulmányok} fejezetben a már a témában létező és általam elemzett kutatásokat mutatom be, azoknak az alkalmazhatóságát az én feladatomnál valamint a megközelítésükben lévő hibákat amelyeket én orvosolni próbálnék.

\textit{A kiberbiztonsági analízis metodológiája} a modellező eszköz részeiről, azok működéséről valamint a használatuk bemutatásával fog foglalkozni.

\textit{A modellező eszköz megvalósítása} az eszközhöz felhasznált technológiákról és az azokban lévő architektúrális megoldásokról, valamint döntésekről szól.

Az \textit{Esettanulmány} című fejezet egy példán való bemutatása az analízis végrehatásának, valamint az eredmények értelmezése

Végül pedig az \textit{Összegzés} alatt lesznek találhatóak az elért célok kiértékelése, alkalmazási lehetőségek, bővítési lehetőségek, valamint a használt források pedig az \textit{Irodalomjegyzékben} lesznek találhatóak.

%%%


%Ezeknek a változásoknak az eredménye tehát az elektronikai megoldások elterjedése a járművekben, ezt a komplex integrált rendszereknél elektromos és elektronikus (electrical and electronic, röviden E/E) architektúrának nevezik. Ez az architektúra rengeteg feladatot kell, hogy tudjon ellátni a korábban említett fejlődés támogatására. Onnan indul a jármű elektronika területe, hogy különböző kényelmi funkciókat 