\chapter{A kiberbiztonsági analízis metodológiája}

A diplomamunkám legfontosabb része a metodológia előállítása volt. A metodológia az ami biztosítja azoknak a céloknak az elérését, hogy az analízis (i) teljes körű legyen, (ii) megismételhető legyen és (iii) már létező információkra építsen, azon túl, hogy az eredménye és használata minél átláthatóbb legyen a stakeholderek és az elemző mérnök számára is.

Ez a metodológia kapcsolja össze az autóiparra jellemző rendszermodelleket a fenyegetésmodellekkel. Ennek segítségére és támogatására készült el a kapcsolódó modellező eszköz és ennek az eredménye az egyik legfontosabb előállított értéke a kiberbiztonsági mérnök feladatkörnek.

Az \textit{Áttekintés} fejezet tartalmaz egy magas szintű végigvezetést a bemenetektől a kimenetig és a közte megtett lépésekről.

A \textit{Termékleírás és fenyegetésmodell származtatása} mutatja be a kiindulómodell elkészítésének lépéseit valamint, hogy abból, hogyan állítjuk elő a fenyegetés modellt.

A \textit{Fenyegetésmodellezés} fejezetben láthatóak a további lépéseket a fenyegetésmodell specifikálása.

A \textit{Támadási fák inicializálása és szerkesztése} fejezet mutatja be a fenyegetésmodellből előállított támadási fák konstrukciójának lépéseit.

A \textit{Dokumentumok generálása és manuális analízis} fejezetben pedig találhatóak a szabvány által előírt output előállítása és az elemző eszköz használatát követő folyamatok.

\section{Áttekintés}



\section{Termékleírás és fenyegetésmodell származtatása}

\subsection{Károkozások attributálása}

\subsection{Értékek attributálása}

\section{Fenyegetésmodellezés}

\subsection{Dependenciák meghatározása}

\section{Támadási fák inicializálása és szerkesztése}

\section{Dokumentumok generálása és manuális analízis}