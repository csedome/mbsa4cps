\chapter{Kapcsolódó tanulmányok}
Ennek a fejezetnek a célja, hogy összefoglaljam a diplomamunkámhoz előzetesen elvégzett kutatómunka során megismert tanulmányok eredményeit, problémáit, valamint a lehetséges alkalmazásukat az én feladatomhoz.

A kutatómunkámat először a fenyegetésmodellezés (threat modeling) területen született tanulmányok olvasásával kezdtem, amelyek segítettek megismerni a fenyegetések felmérése során felmerülő problémákat és megoldásuknak módjait.

Ezek után az autóipari biztonsági elemzések területén végzett munkák segítettek abban, hogy megismerjem milyen komponenseket és azoknak mely attribútumai lesznek használhatóak egy elemzés során.

Végül pedig megismertem a már meglévő kutatásokat amelyek támadási fák (attack trees) generálásáról szólnak, ezek az üzembiztonság területén elterjedt hibafa (fault tree) analízis eszközének adaptációja a kiberbiztonsági terület támogatására.

\section{Fenyegetésmodellezés}

Az első témába illő kutatás a Karahasanovic et al.\cite{Karahasanovic} kutatása volt "Adapting Threat Modelling Methods for the Automotive Industry" címmel. Ez két fenyegetésmodellezési keretrendszert mutat be, egyik az Intel-hez köthető TARA (Threat Agent Risk Assessment), ami nem összekeverendő az azonos rövidítéssel fémjelzett Threat Analysis and Risk Assessment metodológiával. Másik pedig a sokkal ismertebb Microsoft által fejlesztett STRIDE. Az előbbi a grafikus modellezési technikák helyett egy könyvtárakon alapuló fenyegetés elemzést mutat be, ahol három könyvtárat használnak, egyik a lehetséges támadó ágenseket, másik az általuk véghezvihető támadásokat a harmadik pedig a jellemző támadási felületeket gyűjti. A kutatás ezen könyvtárakból határoz meg egy részhalmazt ami az autóipari rendszerek ellen alkalmazható. Utóbbi technika már a támadó-centrikusság helyett inkább szoftver-centrikus, ami egy fehér doboz vizsgálatot tesz lehetővé a rendszeren. Ez a későbbiekben még előforduló Data Flow Diagrammok használatára mutat be egy példát amelyben a szoftver komponensek közti kommunikációt modellezi és tud egy támadási útvonalat végigkövetni. Az előbbi technika gyengesége az, hogy csak magasszinten definiálja a fenyegetéseket ami nem elégséges a védelmi mechanizmusok meghatározására. Utóbbi ezzel szemben alkalmas arra, viszont nem lehet vele rendszer szintű védelmet modellezni.

Ezután olvastam el Ma et al.\cite{Ma} kutatását "Threat Modeling for Automotive Security Analysis" címmel, amelyik a fenyegetésmodellezést egy sokkal gyakorlatibb módon közelíti meg, nem feltétlenül a technikai részekre koncentrál, hanem a termékfejlesztési életciklust és a már meglévő üzembiztonsági analíziseket is figyelembe veszi. Helyesen jelzi a szükségét az analízis szintekre bontásának, hasonlóan az üzembiztonsághoz, azonosítja az igényét egy funkcionális kiberbiztonsági tervnek valamint egy technikai kiberbiztonsági tervnek, kiegészítve egy termékspecifikációval. Ezeknek a jelenlétét pedig lebontja az elsőt a tervezési fázisban, magasszintű követelmények azonosításához, a másodikat a termékfejlesztési szakaszhoz, bemenetként a rendszermodellt használva, a specifikációt pedig a gyártási fázisban való használattal. A kutatás tartalmaz még egy esettanulmányt amely egy jármű utasterének a biztonsági analízisén vezet végig, a Microsoft Threat Modelling Tool használatával, ami a STRIDE keretrendszerre épít, data flow diagrammokat használ és modell alapon generál lehetséges fenyegetéseket. Konklúzióként emeli ki az igényt, hogy a biztonsági analízis modellezési paradigmáit, valahogyan integrálni szeretnék a rendszermodellezés paradigmáival, ezzel biztosítva, hogy az analízis a változások során napra kész maradjon. A termékfejlesztési fázisok és kiberbiztonsági elemzéseket a \ref{fig:MA} ábrán részletezik.

\begin{figure}[!ht]
\centering
\includegraphics[width=150mm, keepaspectratio]{figures/03_MA.png}
\caption{Fenyegetésmodellezés a termékfejlesztési életciklusokban\cite{Ma}}
\label{fig:MA}
\end{figure}

Ezen a területen még egy kutatást \cite{Vivek} vizsgáltam meg ami egy esettanulmány volt egy tetszőleges autóipari komponensre, kiértékeléshez egy módosított STRIDE modellt használt, valamint az ISO 21434-ben definiált kockázatelemzés kezdeti lépéseit amelyek a lehetséges fenyegetések feltárására vannak alkalmazva.

\section{Kiberbiztonság és üzembiztonság kapcsolata}

Ezzel a témával kapcsolatban Dantas et al. \cite{Dantas} kutatását vizsgáltam meg annak területén, hogy a szabványosított kockázatelemzésre, hogyan lehet már meglévő technikákat alkalmazni valamint, hogy a kockázatelemzés, hogyan illeszkedik be már létező folyamatokba. A dokumentum részletesen elemzi a kiberbiztonság fontosságát az autóiparban, valamint a szoftverfrissítés jelenlétét mint fontos eszközt esetleges sérülékenységek javításában. Szintén elemezve van a rendszeres és folyamatos auditálása és kiértékelése ezeknek a folyamatoknak. Ezekhez jelzi a lehetőséget különböző domain-specifikus nyelvek használatának lehetőségét és automatizálás integrálását, illetve modellellenőrző rendszerek bevezetését. Van még szó az üzembiztonság területén alkalmazott FTA és FMEA analízisek technikájának felhasználásáról a támadási útvonalak elemzésében, illetve idéz több más tanulmányt és keretrendszert amelyek szintén ezen alkalmazásokat ösztönzik.

Szintén olvastam Bohner et al.\cite{Bohner} kutatását amely az üzembiztonsági architektúrát terjesztené ki a kiberbiztonsági kockázatok kezelésére. Helyesen hívja fel a figyelmet arra, hogy a kiberbiztonságban használt CIA triádból kettő, az integritás (integrity) valamint a rendelkezésre állás (availability) az üzembiztonság területén is alkalmazva vannak. Említésre kerülnek itt a memória particionálás mint ami mint a két területen csökkentik a kockázatot, az üzenetek védelmét módosítás ellen, valamint összességében a kiberbiztonság mint részhalmaza az üzembiztonságnak ahol a véletlenszerűen előforduló hibák helyett a szándékosan okozott hibákat kell figyelembe vennünk. A kockázat csökkentő intézkedések alkalmazása pedig szignifikánsan tudja mind a két biztonság hatékony szolgáltatását.

Még ami ide tartozna az Chulp et al.\cite{Chulp} kutatása ami egy ThreatGet nevezetű kiberbiztonsági kockázatelemző eszköz működési elvét mutatja be amely a bécsi egyetemen készült. Ez az eszköz gyakorlatiasan írja le a tervezési fázisban elvégzendő kockázatelemzés menetét, ebben már az ASPICE-szal ellentétben láthatunk feedback alkalmazását a folyamat lépései közt, azonban az én célommal ellentétben ez egy külön modellt használ kockázatelemzésre amelyet össze kell hasonlítani majd a meglévő rendszermodellel, ahogy az a \ref{fig:CHULP} ábrán látható.

\begin{figure}[!ht]
\centering
\includegraphics[width=75mm, keepaspectratio]{figures/03_CHULP.png}
\caption{Fenyegetésmodellezés folyamata a tervezési fázisban\cite{Chulp}}
\label{fig:CHULP}
\end{figure}

Chulp et al.\cite{Chulp} kutatásában továbbá hasonlóan az én munkámhoz a STRIDE fenyegetésmodellezési keretrendszert valamint a CIA attribútumait használják. Szintén érdemes kiemelni a fenyegetésmodellezésre jellemzően használt Data Flow Diagram kiegészített formáját amelyet Extended Data Flow Diagramnak neveznek, ebben egy részről kompozíciók modellezését teszik lehetővé, másrészről pedig értékek (asset) megjelenítéséről is gondoskodtak.
Ebben a kutatásban hangzik el először az automatizált támadási fa generálásának fogalma, valamint a tanulmány támadási gráfokat is definiál. A kutatás jól használja fel az Extended Data Flow Diagram rendszermodelljét támadási fák és gráfok generálására amelyekből támadási utakat vezet le amelyek alkalmasak lesznek valódi kockázatok meghatározására.

\section{Támadási fa generálás}

Elsőnek Sowka et al.\cite{Sowka} publikációját olvastam amelyben különböző alkalmazásait értékelte az automatikus támadási fáknak az autóipari kiberbiztonság doménjében. Az írás adott egy általános áttekintést a terület fontosságáról, a szabályozási környezet aktuális helyzetéről majd összehasonlította a különböző elérhető megoldásokat az adott problémára. Ezekből választottam én is további kutatásokat amelyeket érdemes lehet átolvasni.

Salfer et al.\cite{Salfer} által bemutatott módszer egy magas fokú modellezett megoldás alapján való támadási utak előállítását határozza meg. Kifejezetten érdekes, ahogy felépíti a metamodelljét egy rendszer és egy támadó modellnek is. A rendszermodell meghatározza az elektronikus vezérlő egységeket, szoftvereket, kommunikációs hálózatokat és értékeket (\textit{asset}), a támadó modell pedig tartalmazza a tudást, motivációt, amelyek aztán a támadások megvalósíthatóságának értékelésében játszanak szerepet. Szintén elemzi ezeknek a támadási utaknak az alkalmazását a rendszer kiberbiztonsági (penetrációs) tesztelésénél és jól ismeri fel, hogy ez egy magasszintű white-box tesztelésben lehetne felhasználható. 

Karray et al.\cite{Karray} kutatása sokban épít az előző bekezdésben említettre, azonban itt nem lehet egy explicit támadómodellről beszélni. A rendszermodell használ bizonyos tulajdonságokat amelyek jelzik a támadó szükséges tudását vagy belépési szükségét, viszont kevesebb feltevést használ a támadások meghatározásánál. A gráf trnaszformáció valamint a rendszermodell még alkalmas lehet a saját munkámhoz.

Végül pedig Bryans et al.\cite{Bryans} kutatását néztem meg amely kombinálja a modell alapú valamint a könyvtár alapú automatizált generálást, azaz a modell és template alapú támadási fa generálást. A Chulp et al.\cite{Chulp} kritikája alapján is ez volt kiemelve mint legérettebb megoldás, valamint ez is az egyik legfiatalabb. A támadó modell helyett előredefiniált támadási mintákat használ fel, amelyek segítségével rekurzívan tud a fa leveleiből kifejteni komplexebb támadásokat egyfajta bottom-up megközelítésben. A rendszermodell szemben a template-ekkel sokkal egyszerűbb, nem különít el ECU funkcionalitást vagy értékeket, emiatt ez a része nem lesz használható a munkám során, viszont ez teszi lehetővé teszteléskor a black-box megközelítést, valamint akár teszt kódok és eszközök integrációjával is lehetne használni deszkriptív template-ek esetén. Célom az, hogy ezt a fajta kombinált megközelítést alkalmazzam erősebben rendszermodellezett megközelítéssel és egyszerűsítettebb template-ekkel.
