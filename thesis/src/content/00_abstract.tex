\pagenumbering{roman}
\setcounter{page}{1}

\selecthungarian

%----------------------------------------------------------------------------
% Abstract in Hungarian
%----------------------------------------------------------------------------
\chapter*{Kivonat}\addcontentsline{toc}{chapter}{Kivonat}

A modern gépjárművekben egyre jelentősebb szerepet kapnak a számítástechnikai megoldások. Ma egy prémium személyautó közel 150 elektronikus vezérlőegységgel (ECU) és számos kommunikációs sínnel rendelkezik. 

Ezek a feltételek egyrészről lehetővé tették komplexebb üzembiztonsági (safety) megoldások és fejlett vezetéstámogató rendszerek (ADAS) használatát, másrészről viszont a járműbe integrált elektronikai eszközök és azoknak az infokommunikációs hálózatokhoz való csatlakozása megnövelte a lehetséges kiberbiztonsági fenyegetések számát.

Az elmúlt években a járművek ellen elkövetett kibertámadások száma évről évre folyamatosan növekedett és ez a szám hálózati kommunikációban résztvevő járművek terjedésével csak tovább fog növekedni.

Az autóiparban a kockázatalapú biztonság-kezelés terjedt el, mint kiberbiztonsági alapelv, amely a felfedezett fenyegetésekhez, a megállapított kockázat alapján határozza meg az egyes védelmi mechanizmusok szükségességét.

Ezen fenyegetések, kockázatok és védelmi mechanizmusok megállapítására vonatkozó előírások már megtalálhatóak a modern autóipari szabványok közt, viszont az alkalmazásuk még további támogatást igényel. Ebben nyújthatnak segítséget a már elterjedt általános IT biztonsági keretrendszerek, valamint az autóiparban már régóta jelenlévő üzembiztonsági elemzés eszközei.

A feladatom célja, hogy adjak egy metodológiát amely segíti az autóipari rendszerek tervezési fázisban történő kiberbiztonsági analízisét, valamint megvalósítsak egy eszközt ami minnél magasabb szintű automatizálással teszi lehetővé az elemzés elvégzését. 

\vfill
\selectenglish


%----------------------------------------------------------------------------
% Abstract in English
%----------------------------------------------------------------------------
\chapter*{Abstract}\addcontentsline{toc}{chapter}{Abstract}

Electrical and Electronic (E\&E) solutions are playing an increasingly important role in modern automotives. Today, a premium car contains around 150 electronic control units (ECU) and several communication buses.

On the one hand, these conditions enabled the use of more complex safety solutions and advanced driver assistance systems (ADAS), but on the other hand, the electronic devices integrated in the vehicle and their connection to infocommunication networks increased the number of possible cyber security threats.

In recent years, the number of cyber attacks against vehicles has increased year by year, and this number will only continue to increase with the spread of vehicles participating in network communication.

In the automotive industry, risk-based security management has spread as a basic cyber security principle, which determines the need for individual protection mechanisms for discovered threats based on the established risk.

Provisions for establishing these threats, risks and defense mechanisms can already be found in modern automotive industry standards, but their application still requires further support. The general IT security frameworks that are already widespread, as well as the operational safety analysis tools that have been present in the automotive industry for a long time, can help in this.

The goal of my task is to provide a methodology that helps the cyber security analysis of automotive systems in the design phase, as well as to implement a tool that enables the analysis to be carried out with the highest possible level of automation.

\vfill
\cleardoublepage

\selectthesislanguage

\newcounter{romanPage}
\setcounter{romanPage}{\value{page}}
\stepcounter{romanPage}